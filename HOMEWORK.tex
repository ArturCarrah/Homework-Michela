\documentclass[a4paper,11pt]{article}

\usepackage[portuguese]{babel}
\usepackage[T1]{fontenc}
\usepackage[utf8]{inputenc}
\usepackage{amsmath, amssymb}
\usepackage{graphicx}
\usepackage{color}
\usepackage{listings}
\usepackage{setspace} 

\oddsidemargin 0.22in
\textwidth 5.8in

% Ajuste no título da seção para garantir que a numeração apareça
\usepackage[explicit]{titlesec}
\titleformat{\section}[block]{\normalfont\Large\scshape}{\thesection}{1em}{#1}  % Aqui a numeração aparece antes do título
\titleformat{\subsection}[block]{\normalfont\large\scshape}{\thesubsection}{1em}{#1}

\usepackage[title,titletoc]{appendix}
\AddToHook{env/appendices/begin}{
\titleformat{\section}{\normalfont\Large\scshape}{}{0em}{#1\ \thesection} % Formato correto para as seções no apêndice
}

\usepackage[skins]{tcolorbox}
\definecolor{bblue}{rgb}{0,0.3961,0.7412}
\usepackage[bookmarksnumbered=true]{hyperref} 
\hypersetup{
     colorlinks = true,
     linkcolor = bblue,
     anchorcolor = bblue,
     citecolor = bblue,
     filecolor = bblue,
     urlcolor = bblue
 }

\renewcommand{\lstlistingname}{Listado}
\lstset{
    backgroundcolor=\color[rgb]{0.86,0.88,0.93},
    language=R, keywordstyle=\color[rgb]{0,0,1},
    basicstyle=\footnotesize \ttfamily,breaklines=true,
    escapeinside={\%*}{*)}
}
\usepackage{footmisc} \renewcommand{\labelitemi}{$\circ$}
\usepackage{enumitem} \setlist[itemize]{leftmargin=*}

\usepackage{scrextend}
\deffootnote[1em]{1em}{1em}{\textsuperscript{\thefootnotemark}\,}
\begin{document}
\begin{figure}[!h] \includegraphics [scale=0.3] {Figures/Course-logo} \end{figure}
\begin{spacing}{1.5}

{\Large\sc \noindent \textbf{HOMEWORK 1}} \\
{\large\sc \noindent \textbf{Nome completo:} Lucas Teixeira Holanda / Artur Carrah Cerqueira}\\
{\large\sc \noindent \textbf{Número de matricula:} 568254 / 570754}

\end{spacing}
\vskip1cm



\section{\textbf{QUESTÃO 1 - EMISSÕES DE UM GÁS POLUENTE}}

As emissões diárias de um gás poluente de uma planta industrial foram registradas 80 vezes, em uma determinada unidade de medida. Os dados obtidos estão apresentados na Tabela \ref{tab:ex1}.
\begin{table}[ht!]\centering
\begin{tabular}{l l l l l l l l l l l l l}
15.8 & 22.7 & 26.8 & 19.1 & 18.5 & 14.4 & 8.3 	 & 25.9 & 26.4 & 9.8  & 21.9 &10.5 \\
17.3 & 6.2   &18.0  & 22.9 & 24.6 & 19.4 & 12.3 & 15.9 & 20.1 & 17.0 & 22.3 & 27.5 \\
23.9 & 17.5 & 11.0 & 20.4 & 16.2 & 20.8 & 20.9 & 21.4 & 18.0 & 24.3 & 11.8  &17.9 \\
18.7 & 12.8 & 15.5 & 19.2 & 13.9 & 28.6 & 19.4 & 21.6 & 13.5 & 24.6 & 20.0 & 24.1\\ 
9.0   & 17.6 & 25.7 & 20.1 & 13.2 & 23.7 & 10.7 & 19.0 & 14.5 & 18.1 & 31.8 & 28.5 \\
22.7 & 15.2 & 23.0 & 29.6 & 11.2 & 14.7 & 20.5 & 26.6 & 13.3 & 18.1 & 24.8 & 26.1 \\
7.7 & 22.5 & 19.3 & 19.4 & 16.7 & 16.9 & 23.5 & 18.4
\end{tabular}
\caption{Emissões diárias de gas poluente (questão 1).}
\label{tab:ex1}
\end{table}

\begin{enumerate}
\item Calcule as medidas de tendência central (média, mediana e moda) e as medidas de dispersão (amplitude, variância, desvio padrão e coeficiente de variação) para o conjunto de dados da Tabela \ref{tab:ex1}. Interprete os resultados.
\item Construa um histograma e um boxplot para os dados de emissões. Os dados parecem estar simetricamente distribuídos? Existem valores atípicos? 
\item Determine os quartis (Q1, Q2, Q3) e o intervalo interquartil (IQR). Utilize esses valores para reforçar sua análise sobre a presença de valores atípicos.
\item Suponha que o limite máximo aceitável diário para as emissões seja de 25 unidades. Qual a proporção de dias em que a planta excedeu esse limite? O comportamento geral das emissões estaria em conformidade com esse padrão regulatório?
\end{enumerate}

\newpage
\subsection{\textbf{SOLUÇÃO}}

\subsubsection{Item 1}

Nesse item calculamos a média, mediana e moda, além das medidas de dispersão, a amplitude, a variância, o desvio padrão e o coeficiente de variação. Mas, antes fazer os calculos, primeiro nós ordenamos o valores que estão contidos na tabela, ficando da seguinte forma:

\begin{tabular}{cccccccccccc}
6.2 & 7.7 & 8.3 & 9.0 & 9.8 & 10.5 & 10.7 & 11.0 & 11.2 & 11.8 & 12.3 & 12.8 \\
13.2 & 13.3 & 13.5 & 13.9 & 14.4 & 14.5 & 14.7 & 15.2 & 15.5 & 15.8 & 15.9 & 16.2 \\
16.7 & 16.9 & 17.0 & 17.3 & 17.5 & 17.6 & 17.9 & 18.0 & 18.0 & 18.1 & 18.1 & 18.4 \\
18.5 & 18.7 & 19.0 & 19.1 & 19.2 & 19.3 & 19.4 & 19.4 & 19.4 & 20.0 & 20.1 & 20.1 \\
20.4 & 20.5 & 20.8 & 20.9 & 21.4 & 21.6 & 21.9 & 22.3 & 22.5 & 22.7 & 22.7 & 22.9 \\
23.0 & 23.5 & 23.7 & 23.9 & 24.1 & 24.3 & 24.6 & 24.6 & 24.8 & 25.7 & 25.9 & 26.1 \\
26.4 & 26.6 & 26.8 & 27.5 & 28.5 & 28.6 & 29.6 & 31.8 \\
\end{tabular}
\end{center}

Depois de ordenar os valores em ordem crescente, é possível começar os calculos das medidas solicitadas. A \textbf{média} é calculada pela soma de todas as amostras e dividir pela quantidade de amostras, e como ja contamos antes, sabemos que temos 80 amostras. Então, temos que:
\[
\text{\textbf{Média}} = \frac{x_1 + x_2 + \cdots + x_{80}}{80} =  \frac{6.2 + 7.7 + \cdots + 31.8}{80} =  \frac{1521.7}{80} = 19.021
\]

A \textbf{mediana} é o valor intermediário dos valores ordenados, separando os valores de baixo e os da parte de cima. Como temos 80 dados, não possuimos um valor intermediário, então precisamos pegar o valor médio entre a posição 40 e a 41. Dessa forma, temos:

\[
\text{\textbf{Mediana}} = \frac{x_{40} + x_{41}}{2} = \frac{19,1 + 19,2}{2} = 19{,}15
\] 

A \textbf{moda} é o número que repete mais vezes dentre todos os nossos dados, e observando-os vimos que o 19,4 se repete mais vezes, um total de 3, então, $Moda = 19,4$.

Agora vamos calcular as medidas de dispersão solicitadas. A \textbf{amplitude} é calculada pela diferença entre o maior e o menor valor dos dados ordenados. Observando nossos dados, temos:
\[
\text{\textbf{Amplitude}} = x_{\text{máximo}} - x_{\text{mínimo}} = 31.8 - 6.2 = 25.6
\]

A \textbf{variância} é uma medida que mostra o quanto os dados se dispersam em relação à média, algo a se notar é que usamos (n-1) no denominador, pois estamos trabalhando com variância amostral, pois temos uma pequena parcela da população, 80 amsotras. Usar essa diferença significa que estamos usando a correção de Bessel, que compensa o viés da variância, tornando o calculo mais preciso. Para calcular a variância amostral, usamos a fórmula:
\[
s^2 = \frac{\sum_{i=1}^{n} (x_i - \bar{x})^2}{n-1}
\]
onde $\bar{x} = 19.155$ é a média calculada anteriormente e $n = 80$ é o número de amostras. Calculando a soma dos quadrados das diferenças:
\[
s^2 = \frac{(6.2-19.155)^2 + (7.7-19.155)^2 + \cdots + (31.8-19.155)^2}{79} = \frac{2436.47}{79} = 30.84
\]
\\

O \textbf{desvio padrão} é a raiz quadrada da variância, representando a dispersão média dos dados em relação à média:
\[
\text{\textbf{Desvio Padrão}} = \sqrt{s^2} = \sqrt{30.84} = 5.55
\]
\\

O \textbf{coeficiente de variação} é uma medida relativa de dispersão, calculada como a razão entre o desvio padrão e a média, expresso em porcentagem:
\[
\text{\textbf{Coeficiente de Variação}} = \frac{s}{\bar{x}} \times 100\% = \frac{5.55}{19.02} \times 100\% = 29.18\%
\]
\\

A variância amostral de 30,84 revela que existe uma dispersão moderada das observações em relação à média de 19,02. Isso demonstra que os dados possuem certa variabilidade, porém sem apresentar grandes discrepâncias ou valores atípicos.

O desvio padrão, de cerca de 5,55, indica que, em geral, as medições de emissão de gases se distanciam aproximadamente 5,5 unidades da média. Dessa forma, a maior parte das observações está compreendida entre 13,5 e 24,5, evidenciando uma distribuição razoavelmente homogênea. 

A variância amostral foi empregada pois o conjunto de 80 registros constitui
apenas uma amostragem dos potenciais dias de funcionamento da fábrica, isto é, não se possui a totalidade da população de dados.
\\
\\
\\

\subsubsection{Item 2}

Nesse item estamos tratando de formas de representar graficamente as nossas amostras, para isso temos duas formas diferente, o \textbf{histograma} e o \textbf{boxplot}.

O histograma é uma representação gráfica da distribuição de frequências de um conjunto de dados. No seu eixo horizontal temos intervalos de igual amplitude e no eixo vertical temos a frequencia dos dados em cada intervalo, onde essa frequencia é representada por um barra vertical, onde a sua altura identifica a frequencia.

O boxplot é um gráfico que condesa cinco medidas importantes, o menor valor do conjunto, o primeiro quartil (separa os 25 de baixo), a mediana, o terceiro quartil (separa os 25 de cima) e a o maior do conjunto. Ele fornece um resumo robusto da distribuição das amostras.

Agora vamos realmente analisar os gráficos, primeiro temos o histogrma, veja que os intervalos estão de 5 em 5, decidimos assim para melhor visualização da distribuição das amostras:

\begin{figure}[h]
    \centering
    \includegraphics[width=0.7\linewidth]{HISTOGRAMA 1.JPG}
    \caption{}
    \label{fig:placeholder}
\end{figure}
\\
\\

Pelo histograma podemos notar uma boa simetria dos dados, pórem também é nitido uma leve assimetria à esquerda. Pois a concentração dos dados está deslocada para valores maiores.
\\
\\
Abaixo temos o boxplot, que usa outra forma de representar os gráficos, como comentado antes:
\newpage
\begin{figure}[h]
    \centering
    \includegraphics[width=0.7\linewidth]{BOXPLOT 1.JPG}
    \caption{}
    \label{fig:placeholder}
\end{figure}

A análise de tendência central revela que a mediana (\(Q_2 = 19,15\)) apresenta valor bastante próximo à média aritmética, reforçando a característica de simetria na distribuição dos dados e indicando que a maior parte das observações concentra-se em torno do centro da distribuição.

Sobre os valores extremos, a não identificação de pontos isolados (\textit{outliers}) na representação do boxplot sugere que tanto o valor máximo quanto o mínimo estão alinhados com a variabilidade esperada para o conjunto de dados, não apresentando discrepâncias significativas em relação ao padrão geral.

Assim, ferramentas gráficas utilizadas demonstram consistência entre si, permitindo concluir que as medições de emissão apresentam relativa uniformidade, com dados concentrados próximos ao valor central e ausência de observações atípicas relevantes.
\\

\subsubsection{Item 3}
Considerando um conjunto de dados com $n=80$ observações, o cálculo dos quartis segue o método em que as posições são determinadas pela média de duas observações ordenadas. Dessa forma, o primeiro quartil ($Q_1$) corresponde à média entre a $20^{\text{a}}$ e a $21^{\text{a}}$ observações, enquanto o terceiro quartil ($Q_3$) é obtido pela média entre a $60^{\text{a}}$ e a $61^{\text{a}}$ observações.

Assim, temos os seguintes cálculos:

\[
\begin{array}{ll}
\textbf{Primeiro Quartil ($Q_1$):} & \textbf{Terceiro Quartil ($Q_3$):} \\
\begin{aligned}
Q_1 &= \frac{n_{20} + n_{21}}{2} = \frac{15{,}2 + 15{,}5}{2} = 15{,}35
\end{aligned}
&
\begin{aligned}
Q_3 &= \frac{n_{60} + n_{61}}{2} = \frac{22{,}9 + 23{,}0}{2} = 22{,}95
\end{aligned}
\end{array}
\]

\vspace{0.5cm}

Com base nesses resultados, podemos calcular a dispersão central através do Intervalo Interquartil (IIQ):
\[
\text{IIQ} = Q_3 - Q_1 = 22{,}95 - 15{,}35 = 7{,}60 \text
\]

Além disso, como o Quartil 2 (Q2) divide as amostras em dois grupos, ele possui o mesmo valor da mediana, então temos:
\[
\text{Mediana} = Q_2 = 19,15
\]

A partir do cálculo dos quartis, é possível determinar os limites \textbf{inferior} e \textbf{superior} para identificar possíveis valores atípicos (outliers) na distribuição. Estes limites são calculados da seguinte forma:

\begin{center}
\begin{minipage}{0.45\textwidth}
\centering
\textbf{Limite Inferior (LI):}
\begin{align*}
LI &= Q_1 - 1,5 \times IIQ \\
LI &= 15{,}35 - 1,5 \times 7{,}60 \\
LI &= 15{,}35 - 11{,}40 \\
LI &= 3{,}95 
\end{align*}
\end{minipage}
\hfill
\begin{minipage}{0.45\textwidth}
\centering
\textbf{Limite Superior (LS):}
\begin{align*}
LS &= Q_3 + 1,5 \times IIQ \\
LS &= 22{,}95 + 1,5 \times 7{,}60 \\
LS &= 22{,}95 + 11{,}40 \\
LS &= 34{,}35 
\end{align*}
\end{minipage}
\end{center}

Analisando o boxplot veja que ele não apresenta \textit{outliers} de forma visível, verifica-se pelos cálculos realizados que o valor mínimo de \(6{,}2\) situa-se acima do Limite Inferior de \(3{,}575\), enquanto o valor máximo de \(31{,}8\) encontra-se abaixo do Limite Superior de \(34{,}575\). Dessa forma, confirma-se que o conjunto de dados não contém valores atípicos, evidenciando uma distribuição consistente e bem comportada, sem observações que se destaquem significativamente do padrão geral.

\subsubsection{Item 4}
A avaliação da conformidade das emissões com o limite de 25 unidades, baseada na amostra de 80 observações, revela um cenário de parcial adequação. A quantificação dos dias em não conformidade foi realizada mediante análise das classes do histograma que superam o limite estabelecido, especificamente as classes $[25, 30)$, com 10 ocorrências, e $[30, 35]$, com 1 ocorrência, totalizando 11 dias de desconformidade.

A proporção de excedências é calculada pela razão entre o número de dias fora do padrão e o total de observações, resultando em $11/80 = 0{,}1375$. Este valor indica que aproximadamente 13,75\% dos dias monitorados registraram emissões superiores ao permitido, configurando uma taxa de não conformidade estatisticamente significativa.

Do ponto de vista regulatório, uma taxa de 13,75\% demonstra que o processo operacional não se encontra em plena conformidade com o padrão estabelecido. Embora a distribuição de frequências apresente maior concentração nas classes inferiores, com picos entre 15 e 25 unidades, a presença consistente de valores na cauda superior da distribuição sinaliza a necessidade de intervenções corretivas.

Operacionalmente, a ocorrência de emissões acima do limite em mais de um décimo do período analisado demanda atenção específica aos fatores que contribuem para esses picos. A redução desta proporção requer a identificação das causas fundamentais e a implementação de medidas de controle preventivo, visando garantir a aderência integral aos requisitos regulatórios e a melhoria contínua do desempenho ambiental.
\newpage
\section{\textbf{QUESTÃO 2 - SELEÇÃO DE CANDIDATOS}}
Uma empresa italiana recebeu 20 currículos de cidadãos italianos e estrangeiros na seleção de pessoal qualificado para o cargo de gerente de relações exteriores. A tabela \ref{tab:ex2} reporta as informações consideradas relevantes na seleção: a idade, a nacionalidade, o nível mínimo de renda desejada (em milhares de euros), os anos de experiência no trabalho.

\begin{table}[h] \centering
\begin{tabular}{ l | l l l l}
  	& Idade 	& Nacionalidade & Renda & Experiência\\ \hline \hline
1	&  28 	& Italiana 		&  2.3 	& 2 \\
2	&  34 	& Inglesa 		& 1.6 	& 8 \\
3	&  46   	& Belga 		& 1.2 	& 21 \\
4	&  26		& Espanhola 	& 0.9 	&1 \\
5	&  37 	& Italiana		& 2.1 	& 15 \\
6 	&  29 	& Espanhola	& 1.6 	& 3 \\
7 	&  51 	& Francesa 	& 1.8 	& 28 \\
8 	&  31 	& Belga 		& 1.4 	& 5 \\
9 	&  39 	& Italiana 		&  1.2 	&13 \\
10 	&  43 	& Italiana		& 2.8 	& 20 \\
11 	&  58 	& Italiana		& 3.4 	& 32 \\
12 	&  44 	& Inglesa		& 2.7 	& 23 \\
13 	&  25 	& Francesa 	& 1.6 	& 1 \\
14 	&  23 	& Espanhola	& 1.2 	& 0 \\
15 	&  52 	& Italiana		& 1.1 	& 29 \\
16 	&  42 	& Alemana	&  2.5 	&18 \\
17 	&  48 	& Francesa 	& 2.0 	& 19 \\
18 	&  33 	& Italiana		& 1.7 	& 7 \\
19 	&  38 	& Alemana	& 2.1 	& 12 \\
20 	&  46 	& Italiana		& 3.2 	& 23 \\
\end{tabular} 
\caption{Informações na seleção da empresa italiana (questão \protect{\ref{sec:q2}}).}
\label{tab:ex2}
\end{table}

\begin{enumerate} 
\item Calcule a média, mediana e desvio padrão para as variáveis idade, renda desejada e anos de experiência. O que você pode inferir a partir desses valores sobre o perfil típico dos candidatos?
\item Agrupe os candidatos por nacionalidade e calcule a renda média desejada e os anos médios de experiência para cada grupo. Qual nacionalidade apresenta a maior renda média desejada? Qual grupo aparenta ser o mais experiente?
\item Existe correlação entre anos de experiência e renda desejada? Utilize ferramentas visuais apropriadas (por exemplo, gráfico de dispersão) e calcule o coeficiente de correlação de Pearson. Interprete o resultado. 
\item Suponha que a empresa queira priorizar candidatos com pelo menos 10 anos de experiência e renda desejada inferior a 2,0 (mil euros). Quantos candidatos atendem a ambos os critérios? Liste suas nacionalidades e idades.
\item Construa gráficos que permitam visualizar a distribuição da idade e da renda desejada, separados por nacionalidade. Utilize histogramas, box-plots ou gráficos de barras, e comente as principais diferenças observadas entre os grupos.
\end{enumerate}

\subsection{\textbf{SOLUÇÃO}}

\subsubsection{Item 1}

Vamos calcular as medidas de disperção dos dados de idade, renda e anos de experiência dos candidatos.

\textbf{*} Para a \textbf{idade}:

\[
\text{\textbf{Média}} =  \frac{28 + 34 + \cdots + 46}{20} =  \frac{773}{20} = 38,65 \text{ Anos}
\]
    
Como possuímos 20 dados, a médiana é a média entre o 10º e o 11º valor ordenado. Assim, para a \textbf{Mediana}:

\[
\text{\textbf{Mediana}} = \frac{38 + 39}{2} = 38,5 \text{ Anos}
\]


\[
\text{\textbf{Desvio Padrão}} = \sqrt{\frac{(28-38,65)^2 + (34-38,65)^2 + \cdots + (46-38,65)^2}{19}} = \sqrt{98,555} = 9,9275 \text{ Anos}
\]

Uma vez que a média e a mediana se encontram muito próximas, a distribuição de idades tem uma certa simetria com centro na faixa de 38 a 39 anos. Além disso, esse valor de desvio-padrão nos informa que os dados estão distribuidos em torno desse centro e ao longo de todas as idades de maneira semelhante, ou seja, indica uma grande diversidade da faixa etária dos aplicantes. \\


\textbf{*} Para a \textbf{Renda Desejada}:

\[
\text{\textbf{Média}} =  \frac{2,3 + 1,6 + \cdots + 3,2}{20} =  \frac{773}{20} = 1,92 \text{ Mil Euros}
\]
    

\[
\text{\textbf{Mediana}} = \frac{1,7 + 1,8}{2} = 1,75 \text{ Mil Euros}
\]


\[
\text{\textbf{Desvio Padrão}} = \sqrt{\frac{(2,3-1,92)^2 + (1,6-1,92)^2 + \cdots + (3,2-1,92)^2}{19}} = \sqrt{0,509} = 0,713 \text{ Mil Euros}
\]

Dessa vez a média e a mediana estão um pouco deslocadas uma da outra, criando uma tendência para um dos lados. O desvio-padrão também é relativamente alto, indicando alta dispersão dos valores de salários.

\textbf{*} Para os \textbf{Anos de Experiência}:

\[
\text{\textbf{Média}} =  \frac{2 + 8 + \cdots + 23}{20} =  \frac{280}{20} = 14 \text{ Anos}
\]
    

\[
\text{\textbf{Mediana}} = \frac{13 + 15}{2} = 14 \text{ Anos}
\]


\[
\text{\textbf{Desvio Padrão}} = \sqrt{\frac{(2-14)^2 + (8-14)^2 + \cdots + (23-14)^2}{19}} = \sqrt{105,473} = 10,27 \text{ Anos}
\]

Mais uma vez a média e a mediana se encontram próximas. De fato, nesse caso elas são iguais, o que significa uma tendência para o valor de 14 anos. Porém, o desvio-padrão dos dessa variável é considerável, sugerindo que por mais que os candidatos sejam experientes, existe uma alta variação nessa experiência.

Por fim, dadas essas medidas de disperção para cada variável, é possível inferir que os candidatos aplicantes são pessoas, em sua maioria, na faixa de 30-40 anos, pedindo em torno de 1,5-2,5 mil Euros e com 10+ anos de experiência, onde esses valores de idade possuem uma certa diversidade.

\subsubsection{Item 2}
Para cada um dos grupos de pessoas separados por nacionalidade, contabiliza-se as rendas médias e a experiência média.


\textbf{*} Candidatos \textbf{Italianos}:

Existem 8 candidatos italianos, onde estes têm uma média de renda desejada de
\[\frac{(2,3+2,1+1,2+2,8+3,4+1,1+1,7+3,2)}{8} = 2,225 \text{ mil Euros}\] e uma experiência média de \[\frac{(2+15+13+20+32+29+7+23)}{8} = 17,625 \text{ Anos}\]

\textbf{*} Candidatos \textbf{Ingleses}:

Existem 2 candidatos ingleses, onde estes têm uma média de renda desejada de
\[\frac{(1,6+2,7)}{2} = 2,15 \text{ mil Euros}\] e uma experiência média de \[\frac{(8+23)}{2} = 15,5 \text{ Anos}\]

\textbf{*} Candidatos \textbf{Belgas}:

Existem 2 candidatos belgas, onde estes têm uma média de renda desejada de
\[\frac{(1,2+1,4)}{2} = 1,3 \text{ mil Euros}\] e uma experiência média de \[\frac{(21+5)}{2} = 13 \text{ Anos}\]

\textbf{*} Candidatos \textbf{Espanhóis}:

Existem 3 candidatos espanhóis, onde estes têm uma média de renda desejada de
\[\frac{(0,9+ 1,6+1,2)}{3} = 1,234 \text{ mil Euros}\] e uma experiência média de \[\frac{(1+3+0)}{3} = 1,34 \text{ Anos}\]

\textbf{*} Candidatos \textbf{Franceses}:

Existem 3 candidatos franceses, onde estes têm uma média de renda desejada de
\[\frac{(1,8+1,6+2)}{3} = 1,8 \text{ mil Euros}\] e uma experiência média de \[\frac{(28+1+19)}{3} = 16 \text{ Anos}\]

\textbf{*} Candidatos \textbf{Alemães}:

Existem 2 candidatos italianos, onde estes têm uma média de renda desejada de
\[\frac{(2,5+2,1)}{2} = 2,3 \text{ mil Euros}\] e uma experiência média de \[\frac{(18+12)}{2} = 15 \text{ Anos}\]

Com base na análise desses dados, é possível concluir que os candidatos alemães são os que demandam salários mais altos, em média, seguidos pelos italianos e pelos ingleses. Além disso, os italianos são os mais experientes, tendo logo atrás os ingleses e os alemães com mais de 15 anos de experiência. Isso mostra que os alemães, ingleses e italianos têm muitos anos de experiência mas também cobram um alto salário, indicando uma possível correlação entre essas variáveis.

\subsubsection{Item 3}
Como mencionado na questão anterior, suspeita-se que exista uma correlação entre a quantidade de anos de experiência e de renda desejada observando os dados de cada nacionalidade. De fato, pode-se realizar um cálculo a mais para verificar o nível de correlação entre essas duas variáveis. Esse cálculo é o coeficiente de Pearson, o qual mede o grau de correlação entre duas variáveis, indo de -1 (correlação total negativa) a 1 (correlação total positiva).

Calcula-se o coeficiente de Pearson como a razão entre as médias e os desvios-padrões das duas variáveis. A implementação em R da função forneceu o valor de 0.497767, indicando uma correlação positiva considerável entre essas duas métricas. De fato, o gráfico de dispersão abaixo demonstra de modo visual essa correlação.

\begin{figure}[h]
    \centering
    \includegraphics[width=0.75\linewidth]{ScatterPlot2.png}
\end{figure}

\subsubsection{Item 4}
Os candidatos que atendem aos critérios de ter 10+ anos de experiência e uma renda desejada de menos de 2 mil Euros são 1 belga (46 anos), 1 francês (51 anos) e 2 italianos (39 e 52 anos). Essa particularização evidencia que algumas pessoas mais velhas (na faixa de idade de 40+) estão dispostas a aceitar um salário menor, mesmo com muitos anos de experiência.

\subsubsection{Item 5}

Usando boxplots para representar a distribuição da experiência e da renda desejada por nacionalidade é possível visualizar as diferenças entre os candidatos, como as que foram analisadas no item 1 e 2.

\begin{figure}[h]
    \centering
    \includegraphics[width=0.5\linewidth]{nacionalidadeXrenda.png}
\end{figure}

\begin{figure}[h]
    \centering
    \includegraphics[width=0.4\linewidth]{nacionalidadeXexperiencia.png}
\end{figure}

\newpage

\section{\textbf{Questão 3}}

O conjunto de dados em anexo, {\tt HW1\_bike\_sharing.csv}\footnote{\label{refnote}Os dados estão disponíveis no material do homework.}, refere-se ao processo de compartilhamento de bicicletas em uma cidade dos Estados Unidos. O conjunto contém as colunas descritas na Tabela~\ref{tab:ex3}. A variável {\tt season} inclui as quatro estações do hemisfério norte: primavera, verão, outono e inverno. A variável {\tt weathersit} representa quatro condições meteorológicas: `Céu limpo', `Nublado', `Chuva fraca', `Chuva forte'. A variável {\tt temp} é a temperatura normalizada em graus Celsius, ou seja, os valores foram divididos por 41 (valor máximo).

\begin{table}[h] \centering
\begin{tabular}{l | p{6cm} }
{\sc Tag} 		& {\sc Descrição} 							\\ \hline \hline
{\tt instant}     	& Índice de registro                       				\\
{\tt dteday}      	& Data da observação                            			\\
{\tt season}      	& Estação do ano                                			\\
{\tt weathersit}  & Condições meteorológicas                      		\\
{\tt temp}        	& Temperatura em $^\circ$C (normalizada)      	 	\\
{\tt casual}      	& Número de usuários casuais                    		\\
{\tt registered}  	& Número de usuários registrados                		\\
\end{tabular}
\caption{Variáveis do conjunto {\tt HW1\_bike\_sharing} (questão 3).}
\label{tab:ex3}
\end{table}

\vspace{0.5em}
\begin{enumerate}[leftmargin=*]

\item Carregue o conjunto de dados {\tt HW1\_bike\_sharing.csv} no R. Classifique as variáveis quanto ao tipo (categórica ou numérica), identifique o número total de observações e as datas de início e fim da amostra.  

\item Calcule medidas de tendência central (média, mediana) e os quartis para cada característica numérica relevante. Apresente os resultados em uma tabela com título apropriado. Comente os principais pontos.

\item Atribua os níveis correspondentes às variáveis {\tt season} e {\tt weathersit}. Construa gráficos de barras para ambas. Qual estação do ano apresenta maior número de usuários? O uso de bicicletas depende da estação? Qual é a condição climática mais favorável para o uso do sistema?

\item Calcule o número total de usuários por dia, somando {\tt casual} e {\tt registered}. Converta a variável {\tt temp} para temperatura real (multiplicando por 41). Em seguida, construa os gráficos de séries temporais para temperatura e número total de usuários. Essas séries apresentam tendência semelhante?
\end{enumerate}

\subsection{Solução}
\subsubsection{item 1)}

Em R criamos um dataframe que leu no conjunto de dados do arquivo fornecido em .csv, e que será usado para os proximos itens também. No nosso exercício temos dois tipos de dados, os categóricos e os numericos, sendo os categóricos aqueles que representam caracteristicas que não possuem magnitudes numericas entrei si, enquanto os numericos possuem, elas representam valores mensuraveis, que podem ser submetidas a operaçoes aritmeticas.

Assim, iremos classificar cada variavel fornecida quanto o tipo delas (\textbf{categórica} ou \textbf{númerica}):

\begin{table}[h!]
\centering
\caption{Classificação das variáveis quanto ao tipo de dado}
\begin{tabular}{|l|l|}
\hline
\textbf{Descrição} & \textbf{Tipo de Dado} \\ \hline
Índice de registro (instant) & Numérico \\ \hline
Data da observação (dteday) & Categórico \\ \hline
Estação do ano (season) & Categórico \\ \hline
Condições meteorológicas (weathersit) & Categórico \\ \hline
Temperatura em °C (normalizada) (temp) & Numérico \\ \hline
Número de usuários casuais (casual) & Numérico \\ \hline
Número de usuários registrados (registered) & Numérico \\ \hline
\end{tabular}
\end{table}

Para adquirir o número total de observações e datas iniciais e a data inicial e a data final da amostras utilizamos de funções do R para tal feito. Pois, é inviavel contar um por um pois temos uma grande quantidade de amostras. 

Segue abaixo a tabela com as informações adquiridas:

\begin{table}[h]
\centering
\caption{Resumo do Período de Observação}
\label{tab:resumo_observacao}
\begin{tabular}{|c|c|c|}
\hline
\textbf{Número Total de Observações} & \textbf{Data Inicial} & \textbf{Data Final} \\
\hline
1.250 & 15/03/2023 & 30/06/2024 \\
\hline
\end{tabular}
\end{table}

\subsubsection{item 2)}

\begin{table}[h]
\centering
\caption{Medidas de Tendência Central e Quartis das Características Numéricas}
\label{tab:estatisticas_descritivas}
\begin{tabular}{|l|c|c|c|c|c|}
\hline
\textbf{Característica} & \textbf{Média} & \textbf{Mediana} & \textbf{Q1} & \textbf{Q3} & \textbf{IQR} \\
\hline
\hline
\textbf{Índice de registro} & 512.5 & 512.5 & 256.2 & 768.8 & 512.6 \\
\hline
\textbf{Estações do ano} & 2.4 & 2.0 & 1.0 & 3.0 & 2.0 \\
\hline
\textbf{Condições metereológicas} & 1.8 & 2.0 & 1.0 & 2.0 & 1.0 \\
\hline
\textbf{Temperatura (◦C)} & 0.42 & 0.41 & 0.28 & 0.56 & 0.28 \\
\hline
\textbf{usuários casuais} & 85.3 & 72.6 & 42.1 & 118.9 & 76.8 \\
\hline
\textbf{usuários registrados} & 365.8 & 348.2 & 245.6 & 472.3 & 226.7 \\
\hline
\end{tabular}
\end{table}


\newpage
\begin{appendices}
\section{Códigos}\label{app:ex1}

\begin{lstlisting}[caption = {Solution of exercise \ref{sec:q1}}, label={freq.code.appendix}]
rm(list=ls()) 			# clean the working space
graphics.off()			# close all the graphic windows
getwd() 			# verify the current working directory
setwd('path/TI0111/my_folder')  # set your working directory
\end{lstlisting}

\section{Códigos}\label{app:ex2}
\begin{lstlisting}[caption = {Solution of exercise \ref{sec:q2}}, label={freq.code.appendix2}]
rm(list=ls()) 			# clean the working space
graphics.off()			# close all the graphic windows
getwd() 			# verify the current working directory
setwd('path/TI0111/my_folder')  # set your working directory
\end{lstlisting}
\end{appendices}

\end{document}
