\documentclass[a4paper,11pt]{article}

\usepackage[portuguese]{babel} %Caracteres do português
\usepackage[T1]{fontenc}       %Fonte
\usepackage{amsmath, amssymb}  %Fórmulas matemáticas
\usepackage{graphicx}          %Imagens
\usepackage{color}             %Cores nas letras
\usepackage{listings}          %Códigos de linguagens de programação
\usepackage{setspace}          %Espaçamento dos parágrafos
\usepackage{hyperref}
\usepackage{subcaption}        %Para colocar 2 imagens lado a lado

\oddsidemargin 0.22in
\textwidth 5.8in

% Ajuste no título da seção para garantir que a numeração apareça
\usepackage[explicit]{titlesec}
\titleformat{\section}[block]{\normalfont\Large\scshape}{\thesection}{1em}{#1}  % Aqui a numeração aparece antes do título
\titleformat{\subsection}[block]{\normalfont\large\scshape}{\thesubsection}{1em}{#1}

\usepackage[title,titletoc]{appendix}
\AddToHook{env/appendices/begin}{
\titleformat{\section}{\normalfont\Large\scshape}{}{0em}{#1\ \thesection} % Formato correto para as seções no apêndice
}

\usepackage[skins]{tcolorbox}
\definecolor{bblue}{rgb}{0,0.3961,0.7412}
\usepackage[bookmarksnumbered=true]{hyperref} 
\hypersetup{
     colorlinks = true,
     linkcolor = bblue,
     anchorcolor = bblue,
     citecolor = bblue,
     filecolor = bblue,
     urlcolor = bblue
 }

\renewcommand{\lstlistingname}{Listado}
\lstset{
    backgroundcolor=\color[rgb]{0.86,0.88,0.93},
    language=R, keywordstyle=\color[rgb]{0,0,1},
    basicstyle=\footnotesize \ttfamily,breaklines=true,
    escapeinside={\%*}{*)}
}
\usepackage{footmisc} \renewcommand{\labelitemi}{$\circ$}
\usepackage{enumitem} \setlist[itemize]{leftmargin=*}

\usepackage{scrextend}
\deffootnote[1em]{1em}{1em}{\textsuperscript{\thefootnotemark}\,}
\begin{document}
\begin{figure}[!h] \includegraphics [scale=0.3] {Figures/Course-logo} \end{figure}
\begin{spacing}{1.5}

{\Large\sc \noindent \textbf{HOMEWORK 2}} \\
{\large\sc \noindent \textbf{Nome completo:} Lucas Teixeira Holanda / Artur Carrah Cerqueira}\\
{\large\sc \noindent \textbf{Número de matricula:} 568254 / 570754}

\end{spacing}
\vskip1cm



\section{\textbf{QUESTÃO 1 - EMISSÕES DE UM GÁS POLUENTE}}

As emissões diárias de um gás poluente de uma planta industrial foram registradas 80 vezes, em uma determinada unidade de medida. Os dados obtidos estão apresentados na Tabela \ref{tab:ex1}.
\begin{table}[ht!]\centering
\begin{tabular}{l l l l l l l l l l l l l}
15.8 & 22.7 & 26.8 & 19.1 & 18.5 & 14.4 & 8.3 	 & 25.9 & 26.4 & 9.8  & 21.9 &10.5 \\
17.3 & 6.2   &18.0  & 22.9 & 24.6 & 19.4 & 12.3 & 15.9 & 20.1 & 17.0 & 22.3 & 27.5 \\
23.9 & 17.5 & 11.0 & 20.4 & 16.2 & 20.8 & 20.9 & 21.4 & 18.0 & 24.3 & 11.8  &17.9 \\
18.7 & 12.8 & 15.5 & 19.2 & 13.9 & 28.6 & 19.4 & 21.6 & 13.5 & 24.6 & 20.0 & 24.1\\ 
9.0   & 17.6 & 25.7 & 20.1 & 13.2 & 23.7 & 10.7 & 19.0 & 14.5 & 18.1 & 31.8 & 28.5 \\
22.7 & 15.2 & 23.0 & 29.6 & 11.2 & 14.7 & 20.5 & 26.6 & 13.3 & 18.1 & 24.8 & 26.1 \\
7.7 & 22.5 & 19.3 & 19.4 & 16.7 & 16.9 & 23.5 & 18.4
\end{tabular}
\caption{Emissões diárias de gas poluente (questão 1).}
\label{tab:ex1}
\end{table}

\begin{enumerate}
\item Calcule as medidas de tendência central (média, mediana e moda) e as medidas de dispersão (amplitude, variância, desvio padrão e coeficiente de variação) para o conjunto de dados da Tabela \ref{tab:ex1}. Interprete os resultados.
\item Construa um histograma e um boxplot para os dados de emissões. Os dados parecem estar simetricamente distribuídos? Existem valores atípicos? 
\item Determine os quartis (Q1, Q2, Q3) e o intervalo interquartil (IQR). Utilize esses valores para reforçar sua análise sobre a presença de valores atípicos.
\item Suponha que o limite máximo aceitável diário para as emissões seja de 25 unidades. Qual a proporção de dias em que a planta excedeu esse limite? O comportamento geral das emissões estaria em conformidade com esse padrão regulatório?
\end{enumerate}

\newpage
\subsection{\textbf{SOLUÇÃO}}

\subsubsection{Item 1}

Nesse item calcula-se a média, mediana e moda, além das medidas de dispersão, a amplitude, a variância, o desvio padrão e o coeficiente de variação. Mas, antes fazer os calculos, primeiro é necessário ordenar o valores que estão contidos na tabela, ficando da seguinte forma:

\begin{tabular}{cccccccccccc}
6.2 & 7.7 & 8.3 & 9.0 & 9.8 & 10.5 & 10.7 & 11.0 & 11.2 & 11.8 & 12.3 & 12.8 \\
13.2 & 13.3 & 13.5 & 13.9 & 14.4 & 14.5 & 14.7 & 15.2 & 15.5 & 15.8 & 15.9 & 16.2 \\
16.7 & 16.9 & 17.0 & 17.3 & 17.5 & 17.6 & 17.9 & 18.0 & 18.0 & 18.1 & 18.1 & 18.4 \\
18.5 & 18.7 & 19.0 & 19.1 & 19.2 & 19.3 & 19.4 & 19.4 & 19.4 & 20.0 & 20.1 & 20.1 \\
20.4 & 20.5 & 20.8 & 20.9 & 21.4 & 21.6 & 21.9 & 22.3 & 22.5 & 22.7 & 22.7 & 22.9 \\
23.0 & 23.5 & 23.7 & 23.9 & 24.1 & 24.3 & 24.6 & 24.6 & 24.8 & 25.7 & 25.9 & 26.1 \\
26.4 & 26.6 & 26.8 & 27.5 & 28.5 & 28.6 & 29.6 & 31.8 \\
\end{tabular}
\end{center}

Depois de ordenar os valores em ordem crescente, é possível começar os calculos das medidas solicitadas. A \textbf{média} é calculada pela soma de todas as amostras e dividir pela quantidade de amostras, e como já foi dito, existem 80 amostras. Então, tem-se:
\[
\text{\textbf{Média}} = \frac{x_1 + x_2 + \cdots + x_{80}}{80} =  \frac{6.2 + 7.7 + \cdots + 31.8}{80} =  \frac{1521.7}{80} = 19.021
\]

A \textbf{mediana} é o valor intermediário dos valores ordenados, separando os valores de baixo e os da parte de cima. Como existem 80 dados, não existe um valor intermediário, então precisa-se pegar o valor médio entre a posição 40 e a 41. Dessa forma:

\[
\text{\textbf{Mediana}} = \frac{x_{40} + x_{41}}{2} = \frac{19,1 + 19,2}{2} = 19{,}15
\] 

A \textbf{moda} é o número que repete mais vezes dentre todos os dados, e observando-os , é nítido que o 19,4 se repete mais vezes, um total de 3, então, \textbf{Moda $= 19,4$}.

Agora, é feito o cálculo das medidas de dispersão solicitadas. A \textbf{amplitude} é calculada pela diferença entre o maior e o menor valor dos dados ordenados. Observando os dados, tem-se:
\[
\text{\textbf{Amplitude}} = x_{\text{máximo}} - x_{\text{mínimo}} = 31.8 - 6.2 = 25.6
\]

A \textbf{variância} é uma medida que mostra o quanto os dados se dispersam em relação à média, algo a se notar é que tem-se (n-1) no denominador, pois é trabalhado com variância amostral, pois na questão existe apenas uma pequena parcela da população, 80 amostras. Usar essa diferença significa que a correção de Bessel é usada, que compensa o viés da variância, tornando o cálculo mais preciso. Para calcular a variância amostral, usa-se a fórmula:
\[
s^2 = \frac{\sum_{i=1}^{n} (x_i - \bar{x})^2}{n-1}
\]
onde $\bar{x} = 19.155$ é a média calculada anteriormente e $n = 80$ é o número de amostras. Calculando a soma dos quadrados das diferenças:
\[
s^2 = \frac{(6.2-19.155)^2 + (7.7-19.155)^2 + \cdots + (31.8-19.155)^2}{79} = \frac{2436.47}{79} = 30.84
\]
\\

O \textbf{desvio padrão} é a raiz quadrada da variância, representando a dispersão média dos dados em relação à média:
\[
\text{\textbf{Desvio Padrão}} = \sqrt{s^2} = \sqrt{30.84} = 5.55
\]
\\

O \textbf{coeficiente de variação} é uma medida relativa de dispersão, calculada como a razão entre o desvio padrão e a média, expresso em porcentagem:
\[
\text{\textbf{Coeficiente de Variação}} = \frac{s}{\bar{x}} \times 100\% = \frac{5.55}{19.02} \times 100\% = 29.18\%
\]
\\

A variância amostral de 30,84 revela que existe uma dispersão moderada das observações em relação à média de 19,02. Isso demonstra que os dados possuem certa variabilidade, porém sem apresentar grandes discrepâncias ou valores atípicos.

O desvio padrão, de cerca de 5,55, indica que, em geral, as medições de emissão de gases se distanciam aproximadamente 5,5 unidades da média. Dessa forma, a maior parte das observações está compreendida entre 13,5 e 24,5, evidenciando uma distribuição razoavelmente homogênea. 

A variância amostral foi empregada pois o conjunto de 80 registros constitui
apenas uma amostragem dos potenciais dias de funcionamento da fábrica, isto é, não se possui a totalidade da população de dados.
\\
\\
\\

\subsubsection{Item 2}

Esse item trata de formas de representar graficamente as amostras, para isso  duas formas diferentes são usdas, o \textbf{histograma} e o \textbf{boxplot}.

O histograma é uma representação gráfica da distribuição de frequências de um conjunto de dados. No seu eixo horizontal temos intervalos de igual amplitude e no eixo vertical temos a frequencia dos dados em cada intervalo, onde essa frequencia é representada por um barra vertical, onde a sua altura identifica a frequencia.

O boxplot é um gráfico que condesa cinco medidas importantes, o menor valor do conjunto, o primeiro quartil (separa os 25 de baixo), a mediana, o terceiro quartil (separa os 25 de cima) e a o maior do conjunto. Ele fornece um resumo robusto da distribuição das amostras.

Agora analisa-se os gráficos, primeiro no histogrma,é notável que os intervalos estão de 5 em 5, uma forma de melhorar a visualização da distribuição das amostras:

\begin{figure}[h]
    \centering
    \includegraphics[width=0.7\linewidth]{HISTOGRAMA 1.JPG}
    \caption{}
    \label{fig:placeholder}
\end{figure}
\\
\\

Pelo histograma nota-se uma boa simetria dos dados, pórem também é nitido uma leve assimetria à esquerda. Pois a concentração dos dados está deslocada para valores maiores.
\\
\\
Abaixo o boxplot, que usa outra forma de representar os gráficos, como comentado anteriormente:
\newpage
\begin{figure}
    \centering
    \includegraphics[width=0.75\linewidth]{boxplot1.png}
\end{figure}

A análise de tendência central revela que a mediana (\(Q_2 = 19,15\)) apresenta valor bastante próximo à média aritmética, reforçando a característica de simetria na distribuição dos dados e indicando que a maior parte das observações concentra-se em torno do centro da distribuição.

Sobre os valores extremos, a não identificação de pontos isolados (\textit{outliers}) na representação do boxplot sugere que tanto o valor máximo quanto o mínimo estão alinhados com a variabilidade esperada para o conjunto de dados, não apresentando discrepâncias significativas em relação ao padrão geral.

Assim, ferramentas gráficas utilizadas demonstram consistência entre si, permitindo concluir que as medições de emissão apresentam relativa uniformidade, com dados concentrados próximos ao valor central e ausência de observações atípicas relevantes.
\\

\subsubsection{Item 3}
Considerando um conjunto de dados com $n=80$ observações, o cálculo dos quartis segue o método em que as posições são determinadas pela média de duas observações ordenadas. Dessa forma, o primeiro quartil ($Q_1$) corresponde à média entre a $20^{\text{a}}$ e a $21^{\text{a}}$ observações, enquanto o terceiro quartil ($Q_3$) é obtido pela média entre a $60^{\text{a}}$ e a $61^{\text{a}}$ observações.

Assim, tem-se os seguintes cálculos:

\[
\begin{array}{ll}
\textbf{Primeiro Quartil ($Q_1$):} & \textbf{Terceiro Quartil ($Q_3$):} \\
\begin{aligned}
Q_1 &= \frac{n_{20} + n_{21}}{2} = \frac{15{,}2 + 15{,}5}{2} = 15{,}35
\end{aligned}
&
\begin{aligned}
Q_3 &= \frac{n_{60} + n_{61}}{2} = \frac{22{,}9 + 23{,}0}{2} = 22{,}95
\end{aligned}
\end{array}
\]

\vspace{0.5cm}

Com base nesses resultados, é possível calcular a dispersão central através do Intervalo Interquartil (IIQ):
\[
\text{IIQ} = Q_3 - Q_1 = 22{,}95 - 15{,}35 = 7{,}60 \text
\]

Além disso, como o Quartil 2 (Q2) divide as amostras em dois grupos, ele possui o mesmo valor da mediana, então:
\[
\text{Mediana} = Q_2 = 19,15
\]

A partir do cálculo dos quartis, é possível determinar os limites \textbf{inferior} e \textbf{superior} para identificar possíveis valores atípicos (outliers) na distribuição. Estes limites são calculados da seguinte forma:

\begin{center}
\begin{minipage}{0.45\textwidth}
\centering
\textbf{Limite Inferior (LI):}
\begin{align*}
LI &= Q_1 - 1,5 \times IIQ \\
LI &= 15{,}35 - 1,5 \times 7{,}60 \\
LI &= 15{,}35 - 11{,}40 \\
LI &= 3{,}95 
\end{align*}
\end{minipage}
\hfill
\begin{minipage}{0.45\textwidth}
\centering
\textbf{Limite Superior (LS):}
\begin{align*}
LS &= Q_3 + 1,5 \times IIQ \\
LS &= 22{,}95 + 1,5 \times 7{,}60 \\
LS &= 22{,}95 + 11{,}40 \\
LS &= 34{,}35 
\end{align*}
\end{minipage}
\end{center}

Analisando o boxplot veja que ele não apresenta \textit{outliers} de forma visível, verifica-se pelos cálculos realizados que o valor mínimo de \(6{,}2\) situa-se acima do Limite Inferior de \(3{,}575\), enquanto o valor máximo de \(31{,}8\) encontra-se abaixo do Limite Superior de \(34{,}575\). Dessa forma, confirma-se que o conjunto de dados não contém valores atípicos, evidenciando uma distribuição consistente e bem comportada, sem observações que se destaquem significativamente do padrão geral.

\subsubsection{Item 4}
A avaliação da conformidade das emissões com o limite de 25 unidades, baseada na amostra de 80 observações, revela um cenário de parcial adequação. A quantificação dos dias em não conformidade foi realizada mediante análise das classes do histograma que superam o limite estabelecido, especificamente as classes $[25, 30)$, com 10 ocorrências, e $[30, 35]$, com 1 ocorrência, totalizando 11 dias de desconformidade.

A proporção de excedências é calculada pela razão entre o número de dias fora do padrão e o total de observações, resultando em $11/80 = 0{,}1375$. Este valor indica que aproximadamente 13,75\% dos dias monitorados registraram emissões superiores ao permitido, configurando uma taxa de não conformidade estatisticamente significativa.

Do ponto de vista regulatório, uma taxa de 13,75\% demonstra que o processo operacional não se encontra em plena conformidade com o padrão estabelecido. Embora a distribuição de frequências apresente maior concentração nas classes inferiores, com picos entre 15 e 25 unidades, a presença consistente de valores na cauda superior da distribuição sinaliza a necessidade de intervenções corretivas.

Operacionalmente, a ocorrência de emissões acima do limite em mais de um décimo do período analisado demanda atenção específica aos fatores que contribuem para esses picos. A redução desta proporção requer a identificação das causas fundamentais e a implementação de medidas de controle preventivo, visando garantir a aderência integral aos requisitos regulatórios e a melhoria contínua do desempenho ambiental.
\newpage
\section{\textbf{QUESTÃO 2 - PESQUISA ONLINE}}
Um site realiza uma pesquisa online e oferece uma recompensa a um usuário selecionado aleatoriamente que responde a uma série de perguntas. Cada um dos 10 milhões de visitantes diários tem, independentemente, probabilidade $p = 10^{-7}$ de ganhar a recompensa.

\begin{enumerate} 
\item Encontre uma aproximação simples e adequada para a função de massa de probabilidade (PMF) do número de vencedores em um dia, X. Justifique claramente se essa aproximação é apropriada para os valores dados de $n$ e $p$
\item Calcule o valor esperado, E[X], e a variância Var(X), usando tanto a distribuição exata quanto a aproximada. Comente sobre a semelhança entre os resultados
\item Suponha que você ganhe a recompensa, mas que possa haver outros vencedores. Seja $W \sim Pois(1)$ o número de vencedores além de você. Se houver vários vencedores, o prêmio é sorteado aleatoriamente entre todos eles. Encontre a probabilidade de que você realmente receba o prêmio
\item Gere um grande número de simulações diárias para o número de vencedores. Crie uma comparação visual entre os resultados empíricos e a aproximação considerada no item 1. Descreva brevemente o que a visualização indica sobre a qualidade da aproximação
\end{enumerate}

\subsection{\textbf{SOLUÇÃO}}

\subsubsection{Item 1}

Uma aproximação válida para essa premiação é uma distribuição binominal, uma vez que para cada tentativa se escolhe um ou mais ganhadores (ou nenhum). A escolha se justifica, mesmo com os grandes números nos parâmetros, pela natureza da premiação em selecionar pessoas dentre um grande grupo. Os parâmetros dessa distribuição seriam o número de tentativas $n = 10^7$ e a probabilidade de cada pessoa ganhar $p = 10^{-7}$. Portanto, sua PMF pode ser escrita como 

\begin{equation}
P_X(k) = \begin{cases}
\binom{10^7}{k}(10^{-7}) ^k (1-10^{-7})^{10^7-k} , & \text{para } k = 0, 1, 2, \dots \\
0\ , & \text{caso contrário} \\
\end{cases}
\end{equation}

\subsubsection{Item 2}
A expressão correta para a premiação é uma distribuição de Poisson, onde intuitivamente o parâmetro é calculado pelo produto do número de pessoas com a probabilidade de cada uma ganhar, ou seja $\lambda = 10^7\times10^{-7} = 1$. Daí, essa distribuição tem valor esperado e variância igual ao parâmetro $\lambda$. Desse modo, o código em R apresentado a seguir retorna 1 para o valor esperado de ambas distribuições, uma variância de 1 para a distribuição de Poisson e uma variância de 0.9999999 para a distribuição binomial. Portanto, estes valores corroboram com a análise de que a distribuição binomial é uma boa aproximação para  a premiação da pesquisa, uma vez que se encontram muito próximos, de modo que possam ser considerados iguais para todos os efeitos. A distribuição binomial aproximada da figura \ref{fig:distb} é visualmente indistinguível da distribuição de Poisson na figura \ref{fig:distp}, exatamente como se espera a partir dos dados obtidos


\begin{lstlisting}
x <- 0:20
pmf = dpois(x, 1)


pmf2 = dbinom(x,10000000, 0.0000001)


# Valor esperado e Variância

Ex_Poisson = weighted.mean(x, pmf) # O valor esperado é uma média ponderada usando as probabilidades de pesos
cat("Valor Esperado da dist. de Poisson:", Ex_Poisson, "\n")

Ex_Binom = weighted.mean(x, pmf2) 
cat("Valor Esperado da dist. Binomial:", Ex_Binom, "\n")

cat("Variância da dist. de Poisson:", Ex_Poisson, "\n")

cat("Variância da dist. Binomial:", Ex_Binom*(1-0.0000001), "\n")
\end{lstlisting}


\begin{figure}[h]

    \begin{subfigure}{0.5\textwidth}
        \includegraphics[width=0.9\linewidth, height=6cm]{poisson.png} 
        \caption{Gráfico da PMF da distribuição de Poisson}
        \label{fig:distp}
    \end{subfigure}
    \begin{subfigure}{0.5\textwidth}
        \includegraphics[width=0.9\linewidth, height=6cm]{binom.png}
        \caption{Gráfico da PMF da distribuição Binomial}
        \label{fig:distb}
    \end{subfigure}

\caption{Comparação entre a PMF da distribuição aproximada e da real}
\label{fig:image2}
\end{figure}


\subsubsection{Item 3}

A probabilidade de você receber o prêmio mesmo com outras pessoas também ganhando é dada pela Lei da Probabilidade Total. Para você receber dentre um número $i$ de ganhadores, é necessário primeiro ser selecionado para esse grupo e também ganhar dos outros $i-1$ competidores. Portanto, a probabilidade para você ser chamado para o grupo $i$ é $\frac{e^{-1}1^i}{i!}$ e a probabilidade de ganhar é $\frac{1}{i}$. Juntando ambas à Lei da Probabilidade Total: 

\[ \sum_{i=1}^{\infty} \frac{1}{e}\frac{1}{i!\times i} = \frac{1}{e}\sum_{i=1}^{\infty} \frac{1}{i!\times i} = 0.4848291\]

Este resultado é validado pelo seguinte código em R:

\begin{lstlisting}
x <- 0:20
pmf = dpois(x, 1)

# A probabilidade de você ganhar dentre os outros vencedores é dado pela lei da probabilidade total

s = 0
for(i in 1:20){
    s <- s + pmf[i+1]*(1/i)
}

cat("Probabilidade de ganhar:", s)
\end{lstlisting}

\subsubsection{Item 4}

Para uma distribuição de Poisson com 1000000 testes e $\lambda = 1$, a distribuição de probabilidades se comporta exatamente como na aproximação binomial, tendo maiores probabilidades para $k=0$ e $k=1$, fortalecendo a distribuição binomial como aproximação para esse fenômeno.

\begin{figure}[h]
    \centering
    \includegraphics[width=0.5\linewidth]{image.png}
    \caption{Distribuição das probabilidades em uma simulação de Poisson usando um parâmetro de $\lambda = 1$}
    \label{fig:placeholder}
\end{figure}


\newpage

\section{\textbf{Questão 3}}

Você é responsável por monitorar a temperatura de uma CPU multicore em uma unidade
de processamento embarcada. Sob carga normal, a temperatura da CPU apresenta flutuações devido a mudanças na carga de trabalho, nas condições ambientais e na eficiência
do sistema de resfriamento. Testes mostram que a temperatura em regime estacionário
da CPU segue uma distribuição normal com temperatura média µ = 62 ◦C e desviopadrão σ = 3, 5
◦C. Sua tarefa é simular medições de temperatura da CPU e analisar
suas propriedades estatísticas.

\vspace{0.5em}
\begin{enumerate}[leftmargin=*]

\item Crie uma função que gere valores com distribuição normal usando a transformação
de Box--Muller, a partir de entradas aleatórias uniformes. Especificamente:
\begin{enumerate}
  \item[(a)] Gere duas variáveis aleatórias uniformes independentes:
  \[
    U_1, U_2 \sim \mathrm{Unif}(0,1).
  \]

  \item[(b)] Calcule dois valores normais padrão usando as fórmulas de Box--Muller:
  \[
    Z_1 = \sqrt{-2\ln(U_1)}\,\cos(2\pi U_2),
    \qquad
    Z_2 = \sqrt{-2\ln(U_1)}\,\sin(2\pi U_2).
  \]
 \(Z_1\) e \(Z_2\) são variáveis aleatórias independentes com distribuição normal padrão.
Concatene-as para formar um vetor Z de valores normais padrão.

  \item[(c)] Converta cada valor normal padrão para a distribuição de temperatura da CPU:
  \[
    T = 62 + 3.5\,Z.
  \]
\end{enumerate}


\item Use seu gerador de números aleatórios para gerar 1.000 medições de temperatura da CPU. Gere mais 1.000 valores de temperatura utilizando o gerador de números aleatórios normal embutido do R, com a mesma média e desvio-padrão .

\item Para ambos os conjuntos de dados simulados, calcule:
\begin{enumerate}
  \item[(a)] Média amostral.
  \item[(b)] Desvio-padrão amostral.
  \item[(c)] Temperatura mínima e máxima observada.
  \item[(d)] Probabilidade empírica e teórica \(P(T > 68)\).
  \item[(e)] Probabilidade empírica e teórica \(P(60 < T < 65)\).
  \item[(f)] Probabilidade teórica \(P(T > 75)\).
\end{enumerate}
Algum dos conjuntos de dados simulados (1{.}000 amostras) contém valores acima
de \(75^\circ\mathrm{C}\)? Caso não, explique por que eventos raros requerem tamanhos de amostra
grandes para serem observados.

\item Visualize os resultados criando:
\begin{enumerate}
  \item[(a)] Um histograma das temperaturas simuladas da CPU (pode plotar os dois
  conjuntos de dados separadamente ou sobrepostos).
  \item[(b)] A função densidade de probabilidade (PDF) normal teórica (média \(62^\circ\mathrm{C}\),
  desvio-padrão \(3{,}5^\circ\mathrm{C}\)) sobreposta ao histograma.
\end{enumerate}

\item Discuta seus resultados respondendo às seguintes perguntas: As distribuições empíricas da temperatura da CPU se assemelham à curva normal teórica? Quão próximas estão a média amostral e o desvio-padrão amostral dos valores esperados 62 ◦C e 3, 5
◦C?
Há diferenças perceptíveis entre o conjunto de dados gerado com seu RNG manual e
o produzido pelo RNG embutido do R? Como essa simulação pode ajudar na avaliação de estratégias de resfriamento ou de escalonamento dinâmico de clock? Por que
geradores de números aleatórios uniformes são a base dos sistemas de RNG?
\end{enumerate}
\\

\subsection{Solução}
\subsubsection{Item 1 (a)}

Quando se trabalha com medições de temperatura com
distribuição normal, é necessário ter variáveis distribuidas de forma normal, que são variáveis que seguem a distribuição Normal (Gaussiana). Para isso, são criadas duas variáveis aleatórias uniformemente independentes (\(U_1\) e \(U_2\)):

\[
    U_1, U_2 \sim \mathrm{Unif}(0,1).
\]

Uma variável aleatória uniforme é aquela que todos os valores dentro de um intervalo têm a mesma probabilidade de ocorrer.

\subsubsection{Item 1 (b)}
Após a geração das duas variáveis aleatórias independentes e uniformes, usa-se do método de Box-Muller para transformá-las em variáveis aleatórias independentes com distribuição normal. As formúlas de Box-Muller são da forma:

\[
    Z_1 = \sqrt{-2\ln(U_1)}\,\cos(2\pi U_2),
    \qquad
    Z_2 = \sqrt{-2\ln(U_1)}\,\sin(2\pi U_2).
\]

Essas duas variáveis seguem a distribuição normal padrão. E para fins de facilitar a análise, \(Z_1\) e \(Z_2\) são concatenados em um vetor Z, que representa as observações seguindo a distribuição normal com média zero e variância com valor unitário.

\[
    Z_1, Z_2 \sim \mathrm{N}(0,1).
\]

\subsubsection{Item 1 (c)}
Depois de gerar o vetor Z, é necessário convertê-lo para a distribuição de temperatura da CPU: 
  \[
    T = 62 + 3.5Z ;(T = Temperatura, Z = vetor)
  \]

Veja que essa conversão é feita por uma transformação linear, onde \(62\) é a média (\(\mu\)) e \(3{,}5\) é o desvio-padrão (\(\sigma\)):

\[
    T = \mu + \sigma Z 
  \]

Sabe-se que a temperatura em regime estacionário segue uma distribuição normal com os valores acima e que a transformação linear da conversão preserva a normalidade dos dados, apenas deslocando a média de 0 para 62 e alterando os pontos devido ao desvio padrão. Assim, a nova variável aleatória T é da seguinte forma:
\[
    T \sim \mathrm{N}(62,3.5^2).
\]

Abaixo segue a função em R que gera os valores com distribuição normal usando a transformação de Box-Muller. Note que cada parte do código está informando de qual subitem ele é (item1 (a), item1 (b) ou item1 (c)).

\begin{lstlisting}
box_muller_cpu_temp <- function(n) {
  #dados sobre a temperatura (media e desvio padrao)
  mu <- 62
  sigma <- 3.5
  
  #Item 1(a): Gerar U1, U2 ~ Unif(0,1) 
  m <- ceiling(n / 2)
  U1 <- runif(m)
  U2 <- runif(m)

  #Item 1(b): Box-Muller
  Z1 <- sqrt(-2 * log(U1)) * cos(2 * pi * U2)
  Z2 <- sqrt(-2 * log(U1)) * sin(2 * pi * U2)
  Z  <- c(Z1, Z2)[1:n]

  #Item 1(c): Conversao
  T <- mu + sigma * Z
  return(T)
}
\end{lstlisting}

\subsubsection{Item 2}
Usando a função criada nos itens anteriores é possível gerar temperaturas da CPU a partir do método de Box-Muller. Assim, foram geradas 1000 medições dessa temperatura (lembre-se que nesse caso será usado média (\mu\)) = \(62^\circ\mathrm{C}\) e desvio padrão (\sigma\)) = \(3.5^\circ\mathrm{C}\)).

Além disso, também foram gerados 1000 medições usando o geraror de números aleatórios normal do próprio R, passando os mesmos parâmetros. Isso foi feito para fins de comparação, para verificar se os resultados são semelhantes. Segue abaixo o código em R:

\begin{lstlisting}
T_box <- box_muller_cpu_temp(1000)

T_rnorm <- rnorm(n = 1000, mean = 62, sd = 3.5)
\end{lstlisting}

Considere que nesse ambiente a função que gera o Box-Muller já foi criada. Para melhor compreensão do que está acontecendo no código, segue uma imagem com algumas saídas das duas formas de geração de temperaturas (Na primeira linha são valores gerados pela função Box-Muller e na segunda linha valores gerados pelo próprio R):

\begin{figure}[h]
    \centering
    \includegraphics[width=0.8\linewidth]{temps.JPG}
\end{figure}

Ao comparar as medições de temperatura obtidas pelo método de Box--Muller com as
geradas pelo gerador normal embutido do \textsf{R}, verifica-se que ambos os conjuntos são
consistentes com a distribuição normal assumida para o regime estacionário da CPU. Em
ambos os casos, as observações se concentram em torno da média de \(62\,^\circ\mathrm{C}\) e
apresentam dispersão compatível com o desvio-padrão de \(3{,}5\,^\circ\mathrm{C}\). Estatísticas amostrais
(média, desvio-padrão e valores mínimo e máximo) não indicam discrepâncias relevantes
entre os métodos, sugerindo que a geração usando Box--Muller reproduz adequadamente
o comportamento esperado de um gerador de variáveis normais.









\subsubsection{Item 3 (a), (b)}
Tendo gerado os 1000 dados de temperatura da CPU de duas maneiras diferentes(Pelo Box-Muller e diretamente pelo R) é possivel calcular as suas médias amostrais e seus desvios padrões.

A média amostral é a divisão entre todas as temperaturas geradas pela quantidade total de amostras, que nesse caso é 1000. Para o desvio padrão calcula-se a diferença entre cada temperatura e a média amostral, obtendo os desvios. Esses desvios são elevados, e a média deles fornecem a variância do conjunto. Por fim, tira-se a raiz quadrada da variância, resultando no desvio-padrão.

Segue abaixo o código gerador das médias amostrais e dos desvios-padrões reutilizando os valores gerados pela função Box-Muller e pelo gerador de numeros do R:

\begin{lstlisting}
#Funcao Box-Muller
mean_box <- mean(T_box)
sd_box <- sd(T_box)

#Gerador de números do R
mean_rnorm <- mean (T_rnorm)
sd_rnorm <- sd(T_rnorm)
   
\end{lstlisting}

Recapitulando, o Tbox é o vetor com os valores de temperatura gerados pelo Box-Muller e o Trnorm é o vetor com as temperaturas geradas pelo próprio R.


\begin{figure}[h]
    \centering
    \includegraphics[width=0.5\linewidth]{mean.JPG}
    \caption{média amostral}
\end{figure}

\newpage

\begin{figure}[h]
    \centering
    \includegraphics[width=0.5\linewidth]{sd.JPG}
    \caption{desvio-padrão}
\end{figure}

A Figura 2 representa a média amostral do Box-Muller e do gerador do R, já a Figura 3 representa os desvio-padrões, também seguindo a mesma ordem. É nitido a semelhança entre os resultados, com diferença baixas entre os valores das médias e dos desvios padrões, o que indicam proximidade entre os valores gerados pelas duas formas.

\subsubsection{Item 3 (c)}
Além da média e do desvio-padrão, um dado muito importante de se obter são os valores máximos e mínimos de temperatura, pois mostra o quanto os valores gerados estão dispersos para ambos os lados.

Para isso usa-se de uma ferramenta do R. Segue abaixo o código usado:

\begin{lstlisting}
#Funcao Box-Muller
min_box <- min(T_box)
max_box <- max(T_box)

#Gerador de números do R
min_rnorm <- min(T_rnorm)
max_rnorm <- max(T_rnorm)
\end{lstlisting}

As figuras abaixo mostram os resultados obtidos de máximos e mínimos. Na primeira tem-se os máximos de temperatura e na segunda os mínimos.

\begin{figure}[h]
    \centering
    \includegraphics[width=0.5\linewidth]{max.JPG}
    \caption{Temperatura máxima}
\end{figure}

\begin{figure}[h]
    \centering
    \includegraphics[width=0.5\linewidth]{min.JPG}
    \caption{Temperatura mínima}
\end{figure}

Note que na Figura 4 que mostra as temperaturas máximas quanto na Figura 5 que mostra as temperaturas mínimas, a diferença entre as duas funções foi menos de \(1^\circ\mathrm{C}\), o que mostra consistência e proximidade nos dados obtidos.

\newpage
\subsubsection{Item 3 (d),(e),(f)}
Nesse item será feito o calculo das probabilidades empíricas e teóricas de eventos diferentes usando as temperaturas da CPU. Lembre-se que o modelo das temperaturas segue uma distribuição normal com média (\mu\)) = \(62^\circ\mathrm{C}\) e desvio-padrão (\sigma\)) = \(3.5^\circ\mathrm{C}\).

Como se trata de uma distribuição normal , usaremos do resultado da transformação que nos fornece o Z, que seus valores de probabilidade são baseados em uma tabela de relacionada valores de probabilidades fixas (é possível encontrar essa tabela no final do arquivo).

\[
Z = \frac{X - \mu}{\sigma}
\]

Z é o resultado da transformação, \mu\) é a média amostral e \sigma\) é o desvio-padrão.

Além disso, para calcular os valores empíricos e teóricos, é necessário reutilizar os códigos fornecidos dos Itens 1 e 2 e 3 (a)(b)(c). Eles podem ser encontrados no apêndice ao final do arquivo.
\vspace{0.5cm}

\subtitle{\textbf{(d) - P(T > 68):}}
A probabilidade empírica é obtida diretamente a partir dos dados gerados, sendo calculada como a proporção de observações de temperatura que excedem \(68\,^\circ\mathrm{C}\)
em relação ao número total de amostras. Em termos práticos, conta-se quantos valores do conjunto  satisfazem a condição \(T > 68\) e divide-se esse número pelo total de observações.

Já a probabilidade teórica é determinada assumindo que a temperatura da CPU segue uma
distribuição normal com média \(\mu = 62\) e desvio-padrão \(\sigma = 3{,}5\). Nesse caso, a
probabilidade \(P(T > 68)\) é calculada a partir da função de distribuição acumulada da
normal, após a padronização da variável mostrada acima (o valor Z). 

Segue abaixo  o código feito em R e sua saída (Figura 8) com os valores desejados de probabilidades:

\begin{lstlisting}
#valores empiricos
prob_emp_box_D <- mean(T_box > 68)
prob_emp_rnorm_D <- mean(T_rnorm > 68)

#valores teoricos (vale para os dois)
prob_teo_D <- 1 - pnorm(68,mean = 62,sd = 3.5)   
\end{lstlisting}

\begin{figure}[h]
    \centering
    \includegraphics[width=0.5\linewidth]{prob1.JPG}
    \caption{P(T > 68)}
\end{figure}

Utilizando a fórmula da transformação citada anteriormente é possivel calcular o resultado teórico e mostrar que os valores obtidos são extremamente pertos:
\[
Z = \frac{68 - 62}{3{,}5} \approx 1{,}71
\]
Assim, 
\[
P(T > 68) = P(Z > 1{,}71) = 1 - \Phi(1{,}71) \approx 0{,}043
\]

\vspace{0.5cm}

\subtitle{\textbf{(e) - P(60 < T < 65):}}
A probabilidade empírica é estimada a partir dos dados simulados, calculando-se a fração
de observações cuja temperatura está estritamente entre \(60\,^\circ\mathrm{C}\) e
\(65\,^\circ\mathrm{C}\). Esse procedimento consiste em contar o número de amostras que
satisfazem a condição \(60 < T < 65\) e dividir esse total pelo número de medições
realizadas.

Por outro lado, a probabilidade teórica é obtida assumindo que a variável aleatória
\(T\) segue uma distribuição normal com média \(\mu = 62\) e desvio-padrão
\(\sigma = 3{.}5\). Nesse contexto, a probabilidade desejada é expressa como
\[
P(60 < T < 65) = F_T(65) - F_T(60),
\]
onde \(F_T(\cdot)\) representa a função de distribuição acumulada da normal.

Segue abaixo  o código feito em R e sua saída (Figura 9) com os valores desejados de probabilidades:

\begin{lstlisting}
#valores empiricos
prob_emp_box_E <- mean(T_box > 60 & T_box < 65)
prob_emp_rnorm_E <- mean(T_rnorm > 60 & T_rnorm < 65)

#valores teoricos (vale para os dois)
prob_teo_E <- pnorm(65,mean = 62,sd = 3.5) - pnorm (60,mean = 62,sd = 3.5)
\end{lstlisting}

\begin{figure}[h]
    \centering
    \includegraphics[width=0.5\linewidth]{prob2.JPG}
    \caption{P(60 < T < 65)}
\end{figure}

Para o cálculo teórico dessa probabilidade, a variável aleatória \(T\) é
padronizada utilizando a transformação da distribuição normal, substituindo \(\mu = 62\,^\circ\mathrm{C}\) e \(\sigma = 3{,}5\,^\circ\mathrm{C}\), obtêm-se os valores
padronizados aos limites do intervalo:
\[
Z_1 = \frac{60 - 62}{3{,}5} \approx -0{,}57,
\qquad
Z_2 = \frac{65 - 62}{3{,}5} \approx 0{,}86.
\]
Assim,
\[
P(60 < T < 65) = P(-0{,}57 < Z < 0{,}86) = \Phi(0{,}86) - \Phi(-0{,}57) \approx 0{,}52.
\]
\newpage

\vspace{0.5cm}

\subtitle{\textbf{(f) - P(T > 75):}}
Para a probabilidade empírica \(P(T > 75)\), utiliza se o conjunto de dados simulados. Nesse caso, a probabilidade é calculada como a razão entre
o número de observações cuja temperatura excede \(75\,^\circ\mathrm{C}\) e o total de amostras
geradas. Em termos práticos, verifica-se quantos valores do vetor de temperaturas
simuladas satisfazem a condição \(T > 75\) e divide-se esse número pelo total de
1{.}000 medições.

Devido ao fato de \(75\,^\circ\mathrm{C}\) estar a vários desvios-padrão acima da média
\(\mu = 62\,^\circ\mathrm{C}\), a probabilidade teórica associada a esse evento é extremamente
baixa. Assim, é comum que, em amostras de tamanho limitado, como 1{.}000 observações,
nenhum valor ultrapasse esse limiar. Esse resultado empírico reforça a ideia de que
eventos raros exigem tamanhos de amostra significativamente grandes para serem
observados.

Segue abaixo  o código feito em R e sua saída (Figura 10) com os valores desejados de probabilidades:

\begin{lstlisting}
#valores empiricos
p_emp_box_F   <- mean(T_box > 75)
p_emp_rnorm_F <- mean(T_rnorm > 75)

#valores teoricos (vale para os dois)
p_teo_F <- 1 - pnorm(75, mean = 62, sd = 3.5)
\end{lstlisting}

\begin{figure}[h]
    \centering
    \includegraphics[width=0.5\linewidth]{prob3.JPG}
    \caption{P(60 < T < 65)}
\end{figure}

Para o cálculo teórico da probabilidade \(P(T > 75)\), a variável aleatória \(T\) é padronizada
por meio da transformação da normal padrão, substituindo \(\mu = 62\,^\circ\mathrm{C}\), \(\sigma = 3{,}5\,^\circ\mathrm{C}\) e \(T = 75\,^\circ\mathrm{C}\), tem-se:
\[
Z = \frac{75 - 62}{3{,}5} \approx 3{,}71.
\]
Assim,
\[
P(T > 75) = P(Z > 3{,}71) = 1 - \Phi(3{,}71) \approx 0{,}0001.
\]

Note que na Figura 10, os valores de probabilidades empíricas foram zero, isso diz que nenhuma temperatura gerada ultrapassou 75.

\textbf{Algum dos conjuntos de dados simulados (1.000 amostras) contém valores acima
de 75◦C?} Com base nos resultados obtidos, observa-se que nenhum dos conjuntos de dados simulados
com amostras apresentou valores de temperatura acima de \(75\,^\circ\mathrm{C}\). Em
particular, as probabilidades empíricas estimadas a partir dos dados gerados pelo método
de Box--Muller e pelo gerador normal embutido do R foram ambas iguais a zero.

Por outro lado, o cálculo teórico indica que a probabilidade de ocorrência de temperaturas
acima de \(75\,^\circ\mathrm{C}\) é extremamente baixa,
\[
p_{\text{teo}} = P(T > 75) \approx 0{,}0001019.
\]

Esses resultados ilustram que eventos raros, caracterizados por probabilidades muito
pequenas, exigem tamanhos de amostra significativamente maiores para serem observados
empiricamente com frequência apreciável. Em amostras relativamente pequenas, como no
caso de \(1{.}000\) observações, é comum que tais eventos simplesmente não ocorram,
mesmo quando sua probabilidade teórica é não nula.


\subsubsection{Item 4 (a)}
Para melhor analisar os dados obtidos, foram construídos dois histogramas das temperaturas
simuladas da CPU, utilizando tanto os dados gerados pelo método de Box--Muller quanto
aqueles obtidos pelo gerador normal do R. Os histogramas são importantes para 
avaliar a distribuição empírica das observações, evidenciando a concentração dos valores
em torno da média e a dispersão característica do conjunto de dados.

\begin{figure}[h]
    \centering
    \includegraphics[width=0.8\linewidth]{hist1.JPG}
    \caption{Histogramas das temperaturas simuladas da CPU}
\end{figure}

Na Figura 11, o histograma da esquerda representa aquele gerado pelas amostras da função Box-Muller e o da direita é o gerado pelas amostras da função de geração de números do R. Veja que nos dois os valores de concentram perto da média (62), o que está de acordo com o que já foi discutido anteriormente, mostrando assim uma distribuição normal dos valores.

\subsubsection{Item 4 (b)}

Para comparar a distribuição empírica das temperaturas simuladas com o modelo
probabilístico assumido, sobrepõe-se ao histograma a função densidade de probabilidade
(PDF) teórica da normal com média \(\mu = 62\,^\circ\mathrm{C}\) e desvio-padrão
\(\sigma = 3{,}5\,^\circ\mathrm{C}\). Como o histograma é plotado na escala de densidade
(probabilidade), a curva pode ser desenhada diretamente no mesmo gráfico,
permitindo verificar visualmente a proximidade entre as amostras simuladas e a
distribuição normal teórica. Abaixo, a Figura 12 ilustra exatamente a semelhança entre a angulação da curva do PDF e do histograma, sendo na esquerda a da função Box-Muller e na direita a gerada pelas amostras da função de geração de números do R.

\begin{figure}[h]
    \centering
    \includegraphics[width=0.8\linewidth]{hist2.JPG}
    \caption{Histogramas das temperaturas simuladas da CPU com PDF}
\end{figure}

Os códigos dos histogramas e da curva da PDF se encontram no apêncice no final do arquivo, onde também se encontra o código completo do exercício 3.

\subsubsection{Item 5}
A análise dos histogramas e das estatísticas descritivas indica que as distribuições
empíricas das temperaturas simuladas se assemelham de forma consistente à curva
normal teórica assumida para o regime estacionário da CPU. Em ambos os conjuntos de
dados, observa-se uma concentração dos valores em torno da média, bem como uma
dispersão compatível com o formato simétrico característico da distribuição normal.

As médias amostrais obtidas a partir das simulações encontram-se muito próximas do
valor esperado de \(62\,^\circ\mathrm{C}\), enquanto os desvios-padrão amostrais apresentam
valores coerentes com o desvio-padrão teórico de \(3{,}5\,^\circ\mathrm{C}\). Pequenas diferenças
entre os valores amostrais e teóricos são esperadas devido à variabilidade inerente a
amostras finitas, mas não comprometem o modelo probabilístico.

Além disso, não são observadas diferenças perceptíveis entre o conjunto de dados
gerado pelo gerador manual baseado na transformação de Box--Muller e aquele produzido
pelo gerador normal embutido do \textsf{R}. Ambos os métodos resultam em distribuições
estatisticamente semelhantes, indicando que a implementação do gerador manual é
adequada para a geração de variáveis aleatórias normais.

Esse tipo de simulação é útil na avaliação de estratégias de resfriamento e de
escalonamento dinâmico de clock, pois permite estimar a probabilidade de ocorrência de
temperaturas elevadas e analisar o comportamento térmico do sistema sob diferentes
condições, mesmo antes da implementação em hardware real.

Por fim, geradores de números aleatórios uniformes constituem a base dos sistemas de
geração de números aleatórios porque são mais simples de implementar e podem ser
transformados matematicamente para produzir variáveis aleatórias com outras
distribuições, como a normal. Métodos como a transformação de Box--Muller exemplificam
como sequências uniformes podem ser utilizadas para gerar distribuições mais complexas
necessárias em simulações estatísticas e aplicações de engenharia.
























\newpage
\section{\textbf{Apêndice - Códigos em R}}

\subsection{Exercício 1:}

\begin{lstlisting}
   
\end{lstlisting}

\subsection{Exercício 2:}

\begin{lstlisting}

\end{lstlisting}

\subsection{Exercício 3:}

\begin{lstlisting}


\end{lstlisting}

\end{document}

